\documentclass[12pt]{article}

\usepackage{amsmath}

\usepackage{xcolor}


\usepackage{graphicx}

\usepackage{hyperref}

\usepackage{indentfirst}

\usepackage{geometry}

\usepackage{array}

\usepackage{multirow}

\usepackage[utf8]{inputenc}

\setlength{\parskip}{1em}

\title{Petites randonnées plaisir Aix les Bains}

\author{César POLAERT\\Marie RENAULT}

\date{11 Octobre 2023}

\begin{document}

\maketitle

\newpage

\tableofcontents

\newpage

\section{Planning}

\begin{table}
    \centering
    \begin{tabular}{|c|c|c|c|c|}
      \hline
      \textbf{Heure} & \textbf{Jeudi} & \textbf{Vendredi} & \textbf{Samedi} & \textbf{Dimanche} \\
      \hline
      9:00 & Les balcons du semnoz & Le chaos du Chéran & chemin pavé de la figlia & Du local ?\\
      \hline
      10:00 & " & " & " & " \\
      \hline
      11:00 & " & " & " & " \\
      \hline
      12:00 & " & " & " & " \\
      \hline
      13:00 & " & col de la clochette & Lac d'Annecy & " \\
      \hline
      14:00 & " & " & " & " \\
      \hline
      15:00 & " & " & " & Aix-Les-Bains \\
      \hline
      16:00 & " & " & " & " \\
      \hline
      17:00 & " & " & " & " \\
      \hline
      18:00 & " & " & " & " \\
      \hline
    \end{tabular}
  \end{table}

\section{Notre gîte}

\section{A Pied}

\subsection{col de la clochette}
Temps estimé : 1h40

Avant on peut faire le chaos du Chéran

\subsection{col de la frasse}

Temps estimé : 

Marie pas convaincue car pas de boucle

\subsection{point de vue le revard}

Temps trajet estimé : 28 minutes

Si on a une rando dans le coin, ça pourrait être pas mal

\subsection{chemin pavé de la figlia}

Temps estimé : 4h

Faut faire les tours saint Jaques en même temps !

\subsection{la belle étoile}

Temps estimé : 

Marie dit, à creuser parce que c'est quand même vachement beau même si ça ne fait pas une boucle.

César dit : C'est mort : 1h20 de route

\subsection{Mont Clergeon}

Temps estimé : 

Bof bof des deux partis

\subsection{Torrent sierroz}

Temps estimé : 5 minutes

Un petit torrent qui coule au pied de notre gîte, est-ce qu'on l'explore plus loin ? On verra bien. Il semble vouloir nous mener dans le passage entre les deux montagnes

\subsection{Les balcons du semnoz}

Temps estimé : 6h

Le Jeudi !!

\subsection{Lac d'Annecy}

Temps estimé : 

\subsection{Le pont du diable}

Temps estimé :

Le nant de glapigny, c'est une randonnée aquatique et Marie est une poule mouillée.

\subsection{Le chaos du Chéran}

Temps estimé : 2h30

Après on peut faire le plateau de la clochette

\subsection{Ballade équestre}

\section{Autre}

\subsection{La fromagerie}

Temps estimé

\subsection{Aix-les-Bains}

Temps estimé : 

\end{document}